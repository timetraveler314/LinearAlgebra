%!TEX program = xelatex
\documentclass[cn,hazy,blue,14pt,screen]{elegantnote}

\usepackage{linear}

\title{Linear Algebra Done Right 笔记}

\author{timetraveler314}
\institute{PigeonCrowd}

%\version{2.30}
\date{\zhtoday}

\begin{document}

\maketitle

\section{向量空间}

\section{有限维向量空间}

\subsection{张成空间与线性无关}

\begin{theorem}
    \label{thm-2A-p-11}
    (习题11)设 $v_1, \cdots , v_m$ 在 $V$ 中线性无关,且 $w\in V$.

    则有 $v_1,\cdots,v_m$ 线性无关 $\iff$ $w\notin \spn(v_1,\cdots,v_m)$ .
\end{theorem}

\begin{theorem}
    \label{thm-2A-p-14}
    $\star$(习题14) $V$ 是无限维的当且仅当 $V$ 中存在一个向量序列 $v_1,v_2,\cdots$ 使得当 $m$ 是任意正整数时 $v_1,\cdots, v_m$ 都是线性无关的.
\end{theorem}

\begin{note}
    本定理是证明向量空间无限维的常用方法.
\end{note}

\Sep

\begin{problem}
    证明区间 $[0,1]$ 上的所有实值连续函数构成的实向量空间是无限维的.
\end{problem}

\begin{proof}
    构造 
    $$
    f_n(x)=\left\{ 
        \begin{array}{ll}
            x-1/n, & x\geqslant 1/n , \\
            0, & 0 < x < 1/n.
        \end{array} \right.
    $$
    显然 $f_n(x)$ 在 $[0,1]$ 上连续. 下证 $\forall n, f_{n+1} \notin \spn (f_1,\cdots, f_n)$ . 

    假设 $f_{n+1} = a_1f_1 + \cdots + a_nf_n$ ,诸 $a_j \in \fs$ .注意我们有 $f_{n+1}\left(1/n\right) = \dfrac{1}{n} - \dfrac{1}{n+1} \neq 0$.
    但是 $f_1(1/n) = \cdots = f_n(1/n) = 0$ . 由此,只有 $a_1 = \cdots = a_n = 0$.

    此时 $f_{n+1} = 0$ ,产生矛盾. 故 $f_{n+1} \notin \spn (f_1,\cdots,f_n)$. 至此,我们证明了向量序列 $f_1,f_2,\cdots$ 当 $m$ 为任意正整数时 $f_1,\cdots,f_m$ 线性无关.

    由定理\ref{thm-2A-p-14},这个向量空间是无限维的.
\end{proof}

\sep

\begin{problem}
    设 $p_0,\cdots, p_m \in \polyn[\fs]{m}$ 使得对于每个 $j$ 都有 $p_j(2) = 0$. 证明 $p_0,\cdots, p_m$ 在 $\polyn[\fs]{m}$ 中不是线性无关的.
\end{problem}

\begin{proof}
    假设 $p_0,\cdots, p_m$ 线性无关. 根据线性无关组的长度 $\leqslant$ 张成组的长度,且 $1,z,z^2,\cdots,z^m$ 张成 $\polyn[\fs]{m}$,有 $\polyn[\fs]{m}$ 线性无关组长度 $\leqslant m+1$.

    我们构造组 $z,p_0,p_1,\cdots,p_m$ (长度为$m+2$),容易看出其是线性相关的.


    由前面的问题11,我们知道这意味着 $\exists a_j$ (不全为0)$ \in \fs$ 使得 $z=a_0p_0+\cdots+a_mp_m$. 上式中由于 $p_j(2)=0$ ,可以得到 $2=0$ ,矛盾. 故$z\notin \spn(p_0,p_1,\cdots,p_m)$ .

    再次根据问题11可知组 $z,p_0,p_1,\cdots,p_m$ 线性无关,矛盾. 假设不成立.
\end{proof}

\subsection{基}

\begin{theorem}
    \normalfont \textbf{线性无关组可扩充为基}

    \itshape 在有限维向量空间中,每个线性无关的向量组都可以扩充为向量空间的基.
\end{theorem}

\begin{problem}
    证明或给出反例: 若 $v_1,v_2,v_3,v_4$ 是 $V$ 的基,且 $U$ 是 $V$ 的子空间使得 $v_1,v_2 \in U$, $v_3,v_4\notin U$,则 $v_1,v_2$ 是 $U$ 的基.
\end{problem}

\begin{solution}
    命题是错误的. 给出反例:
    
    取 $V = \rs^4$,一组基为 $(1,0,0,0),(0,1,0,0),(0,0,1,1),(0,0,0,1)$ .

    令 $U=\{(x,y,z,0) \in \rs^4 : x,y,z \in \rs\}$,有$U$ 是 $V$ 的子空间且 $v_1,v_2 \in U$, $v_3,v_4\notin U$. 然而, $v_1,v_2$ 不是$U$的基.
    在这里,不难看出 $v_1,v_2,v_3-v_4$是 $U$ 的一组基.
\end{solution}

\begin{remark}
    这道问题的核心在于即使有 $v_3,v_4 \notin U$ ,我们仍然不能保证它们的任何一种线性组合都不在 $U$ 中,这导致单独$v_1,v_2$不足以张成$U$. 
    我们可以修改条件让结论成立. 例如,让 $\spn(v_3,v_4) \cap U = \{0\}$ 就能保证结论成立.
\end{remark}

\begin{problem}
    设 $U$ 和 $W$ 是 $V$ 的子空间使得 $U=V\oplus W$,并设 $u_1,\cdots,u_m$是 $U$ 的基,且 $w_1,\cdots,w_n$是 $W$ 的基. 证明:$u_1,\cdots,u_m,w_1,\cdots,w_n$ 是 $V$ 的基.
\end{problem}

\begin{proof}
    记题设组为 $B$ . 
    
    先来证明 $B$ 在 $V$ 中线性无关. 假设 $\exists a_j, b_k \in \fs$ 使得
    $$
    a_1u_1+\cdots+a_mu_m+b_1w_1+\cdots+b_nw_n=0,
    $$
    则
    $$
    a_1u_1+\cdots+a_mu_m=-(b_1w_1+\cdots+b_nw_n) \in U\cup W = \{0\}.
    $$ (由 $V=U\oplus W$)

    因此 $a_1u_1+\cdots+a_mu_m=b_1w_1+\cdots+b_nw_n=0$.又$u_1,\cdots,u_m$是 $U$ 的基,且 $w_1,\cdots,w_n$是 $W$ 的基,必然有诸 $a_j,b_k=0$. 也即 $B$ 线性无关.

    再证明 $B$ 张成 $V$ .假设 $v\in V$ ,根据 $V=U\oplus W$, $v=u+w$,其中 $u\in U, w\in W$ .因此每个 $v$ 都能写成 $u$和$w$基底的线性组合,故 $V=\spn B$.

    综合两方面,得证.
\end{proof}

\subsection{维数}

\begin{theorem}
    \normalfont \textbf{和空间的维数}

    \itshape 如果 $U_1,U_2$ 是有限维向量空间 $V$ 的子空间,则
    $$
    \dim (U_1+U_2) = \dim U_1 + \dim U_2 - \dim (U_1 \cap U_2).
    $$
    特别地,如果$U_1+U_2$ 是直和,$\dim (U_1+U_2) = \dim U_1 + \dim U_2$.
    \begin{remark}
        这一式子与两个集合的容斥原理相近,但三个集合的容斥原理并不能同样平移,反例: $V=\rs^2,U_1=\{(x,0):x\in\rs\},U_1=\{(0,y):y\in\rs\},U_1=\{(x,x):x\in\rs\}$ .

        猜测这个结果可能因为子空间的封闭性.这导致两个子空间并没有和集合类似的“和运算”,由此可能“覆盖一些维度”.
    \end{remark}
\end{theorem}

\Sep

\begin{problem}
    设 $v_1,\cdots,v_m$ 在 $V$ 中线性无关,并设 $w\in V$. 证明:
    $$
    \dim \spn (v_1+w,\cdots,v_m+w) \geqslant m-1.
    $$
\end{problem}

\begin{proof}
    注意 $v_j-v_1=(v_j+w)-(v_1+w) (2\leqslant j\leqslant m) \in \spn (v_1+w,\cdots,v_m+w).$

    容易证明,各 $v_j-v_1$ 线性无关,这样得到了一个长度为 $m-1$ 的线性无关组. 由线性无关组可以扩充为基,即证.
\end{proof}

\newpage
\section{线性映射}

\subsection{向量空间的线性映射}

\begin{definition}
    \normalfont \textbf{Linear Map, 线性映射}

    \itshape 从 $V$ 到$W$ 的线性映射是具有下列性质的函数 $T:V\to W:$

    \begin{enumerate}
        \item \normalfont \textbf{Additivity, 加性}
        
        \itshape \quad 对于所有 $u,v\in V$ 都有 $T(u+v) = T(u)+T(v)$.
        \item \normalfont \textbf{Homogeneity, 齐性}
        
        \itshape \quad 对于所有 $\lambda\in\fs,v\in V$ 都有 $T(\lambda v) = \lambda T(v)$.
    \end{enumerate}
    \sep

    \begin{remark}
        加性和齐性不互相蕴含.下为两例

        齐性: $\varphi : \rs^2\to\rs,\forall a\in\rs,v\in\rs^2$ 有 $\varphi(av)=a\varphi(v).$

        我们求这样的$\varphi$.考虑模长函数,因为 $\left|av\right|=\left|a\right|\left|v\right|$. 但是这并不完全是齐性,需要想办法消去标量前的绝对值.

        不妨考虑$p=3$时的范数$\varphi((x,y))=\sqrt[3]{x^3+y^3}$ 显然符合条件.

        加性:为简化问题,我们选择域$\cs$或者$\qs [\sqrt{2}]$ 等等.

        给出 $\varphi :\cs \to \cs, \varphi(a+b\Iu) =a$.可以验证不一定有齐性.

        \begin{note}
            不排除实某些域(如$\qs$)中加性蕴含齐性.
        \end{note}
    \end{remark}
\end{definition}

\Sep

\begin{theorem}
    \normalfont \textbf{线性映射把0映射为0}

    \itshape 设 $T$ 是线性映射,则 $T(0)=0$ .
\end{theorem}

\sep

\begin{theorem}
    \label{thm-3.1-expansion-of-linear-map}
    (线性映射的扩张)有限维向量空间 $V$ 的子空间  $U$ 上的线性映射可以扩张成 $V$ 上的线性映射.

    即:$\forall S\in\lm{U}{W},\exists T\in\lm{V}{W} \ s.t.\ \forall u\in U, Tu=Su.$

    \begin{proof}
        令 $\veclist[m]{u}$ 为 $U$ 一组基. 由线性无关组可扩张为基,这可以扩张为 $V$ 的基: 
        $$\veclist[m]{u},v_{m+1},\cdots,v_n.$$

        定义如下的 $T\in \lm{V}{W}$:
        $$
        Tu_i=Su_i,Tv_j=0,1\leqslant i\leqslant m,m+1\leqslant j\leqslant n.
        $$
        又线性映射完全由其在基上的取值确定,这样的 $T$ 唯一存在. 容易验证其符合条件.
    \end{proof}
\end{theorem}

\Sep

\begin{problem}
    设 $T\in \lm{V}{W} $,且$\veclist[m]{v}$ 是 $v$ 中向量组,使得 $\veclist[m]{Tv}$在$W$中线性无关.

    证明: $\veclist[m]{v}$ 线性无关.
\end{problem}

\begin{proof}
    假设 $\exists \numlist[m]{a} \in \fs$ 使
    $$
    0=\svecsum[m]{a}{v}.
    $$
    施以线性映射 $T$:

    $$
    {\color{red}0=T(0)}=T(\svecsum{a}{v})=\svecsum[m]{a}{Tv}
    $$

    这说明 $\numlist[m]{a} = 0$.
\end{proof}

\begin{note}
    注意本题对于线性映射保持0不变的应用.
\end{note}

\sep

\begin{problem}
    设 $V$是有限维的,$\dim V>0$.设$W$是无限维的,证明:$\lm{V}{W}$无限维.
\end{problem}

\begin{proof}
    我们利用定理\ref{thm-2A-p-14}来证明.

    易知存在向量序列$w_1,w_2,\cdots\in W,\forall m, \veclist[m]{w}$线性无关.

    构造 $T_i \in\lm{V}{W}, T(v_1) = w_i$ ($\veclist[m]{v}$是$V$的基).由\ref{thm-3.1-expansion-of-linear-map},每个 $T$都存在(不一定唯一). 容易证明诸$T_i$线性无关.

    再次应用\ref{thm-2A-p-14},$\lm{V}{W}$无限维.
\end{proof}

\end{document}
